\subsection{Description}
%
Amazon Web Services is a subsidiary of Amazon that offers a metered pay-as-you-go cloud computing platforms and APIs to individuals, businesses, and governments. These web services for cloud computing include a range of simple abstract technical infrastructure and distributed computing building blocks and tools. 
%
\subsection{Services needed for CI/CD}
%
AWS offers more than 200 products and services, including computing, storage, database, networking, application services, deployment, management, analytics, machine learning, developer tools, and IoT tools. Among these, the most popular are Amazon Elastic Compute Cloud \textbf{(EC2)}, Amazon Simple Storage Service \textbf{(Amazon S3)}, Amazon Elastic Block Store \textbf{(EBS)}, Amazon Connect, and AWS Lambda. Down below, a brief description of the AWS services related to CI/CD is given\footnote{https://aws.amazon.com/products/developer-tools/?nc2=h\textunderscore ql\textunderscore prod\textunderscore dt}. 

\subsubsection{Amazon Elastic Computing Cloud (EC2)}
%
EC2 enables users to have a virtual cluster of computers accessible at any time via the Internet. It provides the most comprehensive computing platform, with  various options for processor, storage, networking, operating system, and purchase model. EC2 facilitates flexible application deployment by offering a web service that allows users to boot an Amazon Machine Image (AMI) and customize the virtual machine, which Amazon refers to as an "instance," with the software they prefer. The operating systems currently available to use with Amazon EC2 instances include: Amazon Linux, Windows Server 2012, CentOS 6.5 and Debian 7.4.
%

\subsubsection{AWS CodeCommit}
%
AWS CodeCommit is a fully managed source control service that hosts secure Git-based repositories. It makes it easy for teams to collaborate on code in a secure and highly scalable ecosystem. CodeCommit eliminates the need to operate our own source control system or worry about scaling its infrastructure. One can use CodeCommit to securely store anything from source code to binaries, and it works seamlessly with the existing Git tools.  
%

\subsubsection{Amazon Simple Storage Service (Amazon S3)}
%
Amazon Simple Storage Service is an object storage service that offers industry-leading scalability, data availability, security, and performance. 
%

\subsubsection{AWS CodeBuild}
%
AWS CodeBuild is a fully managed continuous integration service that compiles source code, runs tests, and produces software packages that are ready to deploy. CodeBuild provisions, manages, and scales our build servers continuously and processes multiple builds concurrently, so the builds are not left waiting in a queue.  
%

\subsubsection{AWS CodeDeploy}
%
AWS CodeDeploy is a fully managed deployment service that automates software deployments to a variety of computing services such as Amazon EC2, AWS Fargate, AWS Lambda, and on-premises servers. AWS CodeDeploy makes it easier to rapidly release new features, helps avoid downtime during application deployment, and handles the complexity of updating applications. 
%

\subsubsection{AWS CodePipeline}
%
AWS CodePipeline is a fully managed continuous delivery service that helps automate the release pipelines for fast and reliable application and infrastructure updates. CodePipeline automates the build, test, and deploy phases of the release process every time there is a code change, based on the release model we define. We can easily integrate AWS CodePipeline with third-party services such as GitHub or with our own custom plugin.
%

\subsubsection{AWS CloudFormation}
%
AWS CloudFormation gives us an easy way to model a collection of related AWS and third-party resources, provision them quickly and consistently, and manage them throughout their lifecycles, by treating infrastructure as code. A CloudFormation template describes our desired resources and their dependencies so we can launch and configure them together as a stack. We can use a template to create, update, and delete an entire stack as a single unit, as often as we need to, instead of managing resources individually. 
%

%
\subsection{CI-CD Process}
%
The following example uses two separate AWS accounts for development and production.
In summary, the example has the following workflow\footnote{https://aws.amazon.com/blogs/devops/complete-ci-cd-with-aws-codecommit-aws-codebuild-aws-codedeploy-and-aws-codepipeline/}:

\begin{enumerate}
    \item A change or commit to the code in the CodeCommit application repository triggers CodePipeline with the help of a CloudWatch event.
    
    \item The pipeline downloads the code from the CodeCommit repository, initiates the Build and Test action using CodeBuild, and securely saves the built artifact on the S3 bucket.
    
    \item If the preceding step is successful, the pipeline triggers the Deploy in Dev action using CodeDeploy and deploys the app in dev account.
    
    \item If successful, the pipeline triggers the Deploy in Prod action using CodeDeploy and deploys the app in the prod account.
\end{enumerate}

The following diagram illustrates the workflow:

\begin{figure}[h]
    \centering
    \includegraphics[width=\textwidth]{images/aws_cicd.png}
    \caption{Workflow of CI/CD pipeline in AWS.}
    \label{fig:aws_cicd}
\end{figure}

%

\subsection{Estimated Cost}
%
The total cost of running a CI/CD pipeline on AWS depends on the AWS services used in your pipeline. Monthly charges will vary on your configuration and usage of each product, but if we use all the AWS services mentioned above and accept the default configurations, we can expect to be billed around \$15 per month.
%

\subsection{Advantages}
%
Companies and individuals prefer AWS as their cloud provider because of the numerous AWS benefits it provides.

\begin{itemize}
    \item \textbf{User-friendly:} AWS is easy to use as the platform is specially designed for quick and secure access.There won’t be any problem for a new applicant as well as for an existing applicant.
    \item \textbf{Cost-effective:} AWS offers a pay-as-you-go pricing method, which means that a company will only pay for the services that it needs and has used for a period of time.
    \item \textbf{Scalable and Elastic:} AWS is scalable because the AWS Auto Scaling service automatically increases the capacity of constrained resources as per requirements so that the application is always available. If you use fewer resources and you don’t need the rest of them, then AWS itself shrinks the resources to fit your requirement.
    \item \textbf{Scope of operations: } AWS has a huge and growing array of available services, as well as the most comprehensive network of worldwide data centers.
    
\end{itemize}
The Gartner report summed it up, saying, “AWS is the most mature, enterprise-ready provider, with the deepest capabilities for governing a large number of users and resources.”


%


\subsection{Disadvantages}
These are the limitations of Amazon Web Services:
\begin{itemize}
    \item \textbf{Complex Price Plans:} Amazon’s big weakness relates to cost. While AWS regularly lowers its prices, many enterprises find it difficult to understand the company’s cost structure and to manage those costs effectively when running a high volume of workloads on the service\footnote{https://www.datamation.com/cloud/aws-vs-azure-vs-google-cloud/}.
    \item \textbf{Limitations OF Amazon EC2:} AWS sets default limits on resources which vary from region to region. These resources consist of images, volumes, and snapshots. You can launch the limited number of instance per area. It also provides limited information for the resources managed by Amazon EC2and Amazon VPC console\footnote{https://data-flair.training/blogs/aws-advantages/}. 
    \item \textbf{Technical Support Fee:} AWS charges you for immediate support and you can opt for any packages among 3 which are- \textbf{ Developer}, \textbf{Business}, \textbf{Enterprise}
    \item \textbf{Complex Organisation:} AWS can be overwhelming with the vast amount of services they provide. It can be tough to pick the right option to fit the goal you're trying to accomplish

\end{itemize}





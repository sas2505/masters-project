CI-CD means continuous integration and continuous delivery or deployment. This process is trigered when a code is commited into the version control system. A build tool is used on the CI-CD to create the artificates. The next step is to run some unit and integration tests to find out any bugs. Afterwards, passing all the tests the new version of the software is deployed on the desired environment. Otherwise The developers will work on the bugs and fixing those until it is on deployable states. By this way when the final relaese date will come the software can be released smoothly and faster.

Now-a-days, CI-CD can be setup on cloud environemnt which incorporates the main process of CI-CD (build, test, delivery and release) and additionally provides some features flexibility for the CI-CD. One of the main upgrade is software development and release team do not need to worry about the environemnt specification and different tools setup on that environment as well.

In software release there are some manual processes are used. Some processes are manually configure the software dependency on the production environemnt, deploying the software on the invironemnt as well. This processes are more complicated due to sparse environment releated issues and so on. Furthermore, this processes often indicate missing bugs in software which are found in the phase which is rather risky to fix at that time. And doing this rush also introduce more bugs into the system and in the end team members have to work on a lot of stess and pressure as well. But if these processes are made automate in the very first of phases than these kind of issues can be reduced a lot in the begining. And thus CI-CD becomes a growing and interesting concept in the software release process. It is integrate from the begining so whenever a commit is made by the developer the build is done and then test cases are executed and based on the results it was decided whether this particular version will be released or not. if it was not ok then developer should work on the fix and afterwards the fix will be committed. The CI will do the process again and then upon success of the CI it will deplyed on the desired environemnt as well. This automate deployment in the environment is called CD. Using these two methodologies, CI-CD, together make the system more smooth, bug free and moreover stress free software release process. Furthermore, using CI-CD on cloud gives additional edges for the team reagrding software release steps. Especially, setting up the environemnt and scalling the projects are more flexible and easier. In this report we will dive into more in the CI-CD steps and different tools structure.